\documentclass[a4paper,12pt]{article}
\usepackage{longtable}

\title{Afstudeerlogboek}
\author{Seger Sars}
\date{2025}

\begin{document}

\maketitle

\newpage

\section*{Week 1}

\subsection*{Voortgang}

\begin{longtable}{|l|l|p{0.09\textwidth}|p{0.7\textwidth}|}
\hline
\textbf{Dag} & \textbf{Datum} & \textbf{Aantal uren} & \textbf{Beschrijving werkzaamheden} \\
\hline
\endfirsthead
\hline
\textbf{Dag} & \textbf{Datum} & \textbf{Aantal uren} & \textbf{Beschrijving werkzaamheden} \\
\hline
\endhead
\hline
\endfoot
\endlastfoot
% Add your data rows below, for example:
Maandag   & 03-02-2025 & 8 & Introductie presentaties bijgewoont, met andere stagiairs en medewerkers kennis gemaakt, laptop geinstalleerd en bestaande hardware bekeken. \\ \hline
Dinsdag   & 04-02-2025 & 8 & Veel aan het PoA gewerkt, document gelezen over de begeleider zijn visie van het project en vragen samengesteld voor verduidelijking. \\ \hline
Woensdag  & 05-02-2025 & 8 & Mijn gedeelten van het PoA vrijwel afgewerkt voor de eerste review, buiten dat een goed gesprek gehad met onze begeleider en is de opdracht een stuk duidelijker geworden. \\ \hline
Donderdag & 06-02-2025 & 8 & Werkzaamheden Z \\ \hline
Vrijdag   & 07-02-2025 & 8 & Werkzaamheden Z \\ \hline

\hline

% Continue adding rows as needed
\end{longtable}

\subsection*{Reflectie}

Zie de instructie op pagina 2


\newpage

\subsection*{Instructie}

\subsubsection*{Voortgang}
Door het gestructureerd bijhouden van je werkzaamheden en de voortgang van het project laat je zien dat je de competentie managen beheerst. Daarnaast geeft het weekoverzicht jouw docentbegeleider een inkijkje in jouw werkzaamheden en voortgang gedurende je stage.

\subsubsection*{Reflectie}
Door te reflecteren laat je zien dat je kritisch kunt kijken naar je eigen functioneren en naar het bedrijf waar je stage loopt. Je formuleert verbeterpunten en laat daarmee zien dat je de competentie professionaliseren beheerst.

Je kiest elke week een situatie of onderwerp om op te reflecteren. Je mag daarbij gebruik maken van de STARR-methode, maar dat hoeft niet. De inhoud van je reflectie is belangrijker dan de vorm.

\begin{longtable}{|l|p{0.8\textwidth}|}
\hline
% Add your data rows below, for example:
Situatie:   & Wat was de situatie? Wat moet een lezer weten om de rest van de reflectie te kunnen begrijpen? \\ \hline
Taak:   & Wat was volgens jou je taak? Wat wilde je bereiken?  \\ \hline
Actie:  & Wat heb je concreet gedaan? Op welke andere manier had je de taak ook kunnen volbrengen? \\ \hline
Resultaat: & Wat is het resultaat van jouw actie? \\ \hline
Reflectie:   & Wat is er goed gegaan? Wat is er niet goed gegaan? Wat was jouw rol hierin? Waar ben je tevreden over? Wat zou je kunnen verbeteren? Welk resultaat zou een andere actie opgeleverd hebben? Benoem hierbij ook concrete verbeterpunten. \\ \hline

\hline

% Continue adding rows as needed
\end{longtable}

\subsection*{Reflectie-onderwerpen}
Hieronder vind je een lijst van voorbeeldonderwerpen. Je kunt wekelijks een onderwerp uit deze lijst kiezen, maar je kunt uiteraard ook een eigen onderwerp uitwerken.

\begin{itemize}
    \item Welke eerste indruk hebben collega’s van jou en waarom? Wat kun je hiervan leren voor de toekomst?
    \item Tegen welke uitdagingen ben je deze week aangelopen? Hoe heb je dat aangepakt? Wat is het resultaat? Wat heb je hiervan geleerd?
    \item Waar ben je deze week tevreden over en waarom? Wat kun je hiervan leren voor de toekomst?
    \item Lig je nog op planning? Of is het nodig om deze bij te stellen? Hoe komt dit?
    \item Werk je nog steeds toe naar de projectdoelstelling uit je PvA? Of moet deze aangepast worden?
    \item Zou je bij dit bedrijf willen werken? Waarom?
    \item Wanneer vind je iemand een goede collega? Welke van die eigenschappen herken je in jezelf en je begeleider?
    \item Hoe divers is dit bedrijf (denk aan bv. leeftijd, geslacht, afkomst, etc.)? Wat betekent dat voor de dagelijkse gang van zaken en voor jouw functioneren? Wat zou hierin ideaal zijn en waarom?
    \item Hoe duurzaam is het bedrijf? In welke opzichten kan het bedrijf hierin verbeteren?
    \item Als je dit project over mocht doen, wat zou je dan anders aanpakken en waarom?
\end{itemize}

\end{document}
