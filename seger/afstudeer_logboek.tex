\documentclass[a4paper,12pt]{article}
\usepackage{longtable}

\title{Afstudeerlogboek}
\author{Seger Sars}
\date{2025}

\begin{document}

\maketitle

\newpage

\section*{Week 1}

\subsection*{Voortgang}
\begin{longtable}{|l|l|p{0.09\textwidth}|p{0.7\textwidth}|}
\hline
\textbf{Dag} & \textbf{Datum} & \textbf{Aantal uren} & \textbf{Beschrijving werkzaamheden} \\
\hline
\endfirsthead
\hline
\textbf{Dag} & \textbf{Datum} & \textbf{Aantal uren} & \textbf{Beschrijving werkzaamheden} \\
\hline
\endhead
\hline
\endfoot
\endlastfoot
% Add your data rows below, for example:
Maandag   & 03-02-2025 & 8 & Introductie presentaties bijgewoont, met andere stagiairs en medewerkers kennis gemaakt, laptop geinstalleerd en bestaande hardware bekeken. \\ \hline
Dinsdag   & 04-02-2025 & 8 & Veel aan het PoA gewerkt, document gelezen over de begeleider zijn visie van het project en vragen samengesteld voor verduidelijking. \\ \hline
Woensdag  & 05-02-2025 & 8 & Mijn gedeelten van het PoA vrijwel afgewerkt voor de eerste review, buiten dat een goed gesprek gehad met onze begeleider en is de opdracht een stuk duidelijker geworden. \\ \hline
Donderdag & 06-02-2025 & 8 & Wouter vandaag ziek thuis. Ik ben vandaag aan de slag geweest met het requirements document. Veel functionele requirements al weten te maken, maar de technische eisen blijven nog vaag. \\ \hline
Vrijdag   & 07-02-2025 & 8 & Vandaag begonnen met mijn onderzoeks logboek, omdat ik zeer benieuwd was naar een belangrijk onderwerp voor mijn product. Ik had geen zin om nog meer aan de requirements te werken dus heb ik de code van de vorige stagiairs licht onderzocht en de huidige mechanische hardware getest door het programma aan te passen met een simpele demo functie waarmee het platform heen en weer gaat. De code is slechter dan verwacht. \\ \hline

\hline

% Continue adding rows as needed
\end{longtable}

\newpage

\subsection*{Reflectie}
\noindent\textbf{Hoe divers is dit bedrijf (denk aan bv. leeftijd, geslacht, afkomst, etc.)?
Wat betekent dat voor de dagelijkse gang van zaken en voor jouw func-
tioneren? Wat zou hierin ideaal zijn en waarom?}
\newline\newline
Na mijn eerste week bij Topic valt op dat er een veel diversiteit is binnen het bedrijf.
Tijdens de presentatie op de eerste dag is er voorbij gekomen dat 55\% Nederlands praat en dat er 21 verschillende nationaliteiten zijn binnen Topic.
21 nationaliteiten is best divers voor het aantal medewerkers, al is dit zeker niet egaal verdeelt.
Ook lijkt het erop dat het meeste van de diversiteit in de engineers zit, de meeste van HR die ik heb gezien zijn Nederlands, meeste van het hogere management wat ik heb gezien of ik over heb gehoord is Nederlands, ook de mensen bij finance zijn Nederlands.
Ik weet niet zeker waarom er zoveel diversiteit zit in de engineers, ik denk dat het komt door de grote vraag naar engineers in Nederland.
Door het tekort in werknemers zijn er veel mensen van andere landen gewild om hier te komen werken.
\newline\newline
\noindent\textbf{Wat betekent dit voor het dagelijks gang van zaken en voor mijn functioneren?}
Omdat ik zelf tussen de engineers zit hoor ik eigenlijk vrijwel alleen Engels.
Mijn stage begeleider is Nederlands en dat is toch wel echt fijn.
Mijn Engels is niet slect, ik kan mezelf altijd duidelijk maken, vooral op technisch gebied.
Maar ik denk dat het voor de bedrijfs sfeer een interressant effect kan hebben.
Aan de ene kant is het moeilijker om te grappen en te kletsen met collegas als ze niet dezelfde moedertaal spreken als ik, hierdoor wordt je misschien minder snel hecht of bekend met elkaar dan als je dezelfde taal spreek.
Aan de andere kant is het ook wel een interessante dynamiek, veel van de mensen kunnen niet hun eigen taal spreken en ik denk dat het moeten praten van Engels in plaats van de taal waar je het meest comfortabel in bent, voor een gedeelde ervaring zorgt tussen de niet Nederlandse mensen.
Ik zie ook vaak dat de Nederlanders met de Nederlanders bij het koffie apparaat praten en de andere kant daarvan.

\noindent\textbf{Conclusie}\\
Alles wat ik schriftelijk deed was al in het engels en ik dacht al niet tussen alleen maar Nederlandse engineers te werken voor de rest van mijn leven, maar toch had ik het niet zoals dit verwacht.
Voor een bedrijf wat relatief klein is (tegenover ASML) zijn er naar mijn mening veel niet Nederlandse mensen.
Veel van wat ik heb opgeschreven is mijn persoonlijk beeld na hier een week gelopen te hebben en heb het misschien helemaal bij het verkeerde eind.
Kan goed zijn dat als ik in een Engels team zou zitten, ik binnen een week gewend zou zijn en dikke vrienden ben. 
Dat zal later in mijn carriere moeten blijken.
Voor nu is het een interessant reflectie onderwerp en iets om over na te denken wanneer ik later een bedrijf uitzoek om te gaan werken.


\newpage

\subsection*{Instructie}

\subsubsection*{Voortgang}
Door het gestructureerd bijhouden van je werkzaamheden en de voortgang van het project laat je zien dat je de competentie managen beheerst. Daarnaast geeft het weekoverzicht jouw docentbegeleider een inkijkje in jouw werkzaamheden en voortgang gedurende je stage.

\subsubsection*{Reflectie}
Door te reflecteren laat je zien dat je kritisch kunt kijken naar je eigen functioneren en naar het bedrijf waar je stage loopt. Je formuleert verbeterpunten en laat daarmee zien dat je de competentie professionaliseren beheerst.

Je kiest elke week een situatie of onderwerp om op te reflecteren. Je mag daarbij gebruik maken van de STARR-methode, maar dat hoeft niet. De inhoud van je reflectie is belangrijker dan de vorm.

\begin{longtable}{|l|p{0.8\textwidth}|}
\hline
% Add your data rows below, for example:
Situatie:   & Wat was de situatie? Wat moet een lezer weten om de rest van de reflectie te kunnen begrijpen? \\ \hline
Taak:   & Wat was volgens jou je taak? Wat wilde je bereiken?  \\ \hline
Actie:  & Wat heb je concreet gedaan? Op welke andere manier had je de taak ook kunnen volbrengen? \\ \hline
Resultaat: & Wat is het resultaat van jouw actie? \\ \hline
Reflectie:   & Wat is er goed gegaan? Wat is er niet goed gegaan? Wat was jouw rol hierin? Waar ben je tevreden over? Wat zou je kunnen verbeteren? Welk resultaat zou een andere actie opgeleverd hebben? Benoem hierbij ook concrete verbeterpunten. \\ \hline

\hline

% Continue adding rows as needed
\end{longtable}

\subsection*{Reflectie-onderwerpen}
Hieronder vind je een lijst van voorbeeldonderwerpen. Je kunt wekelijks een onderwerp uit deze lijst kiezen, maar je kunt uiteraard ook een eigen onderwerp uitwerken.

\begin{itemize}
    \item Welke eerste indruk hebben collega’s van jou en waarom? Wat kun je hiervan leren voor de toekomst?
    \item Tegen welke uitdagingen ben je deze week aangelopen? Hoe heb je dat aangepakt? Wat is het resultaat? Wat heb je hiervan geleerd?
    \item Waar ben je deze week tevreden over en waarom? Wat kun je hiervan leren voor de toekomst?
    \item Lig je nog op planning? Of is het nodig om deze bij te stellen? Hoe komt dit?
    \item Werk je nog steeds toe naar de projectdoelstelling uit je PvA? Of moet deze aangepast worden?
    \item Zou je bij dit bedrijf willen werken? Waarom?
    \item Wanneer vind je iemand een goede collega? Welke van die eigenschappen herken je in jezelf en je begeleider?
    \item Hoe divers is dit bedrijf (denk aan bv. leeftijd, geslacht, afkomst, etc.)? Wat betekent dat voor de dagelijkse gang van zaken en voor jouw functioneren? Wat zou hierin ideaal zijn en waarom?
    \item Hoe duurzaam is het bedrijf? In welke opzichten kan het bedrijf hierin verbeteren?
    \item Als je dit project over mocht doen, wat zou je dan anders aanpakken en waarom?
\end{itemize}

\end{document}
