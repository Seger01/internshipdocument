\documentclass{article}
\usepackage{geometry}
\usepackage{hyperref}
\usepackage{enumitem}

\geometry{a4paper, margin=1in}

\title{Plan of Attack}
\author{Seger Sars and Wouter Boerenkamps}
\date{04-02-2025}

\begin{document}

\maketitle

\newpage

\noindent \textbf{Client:}\\
Topic Embedded Systems\\
Engineering department\\
Materiaalweg 4, 5681 RJ Best\\
+31 499336979\\

\vspace{1em}

\noindent \textbf{Company Supervisor:}\\
Dirk van den Heuvel\\
dirk.van.den.heuvel@topic.nl\\
Position: Product Manager and Principal Consultant at TOPIC Embedded Systems\\
\\
and\\
\\
Britta Claes\\
britta.claes@topic.nl\\
Position: Business Manager, People Manager, and Account Manager

\vspace{1em}

\noindent \textbf{Educational Institution:}\\
Avans University of Applied Sciences\\
Academy for Industry and Informatics\\
Onderwijsboulevard 215\\
5223 DE 's-Hertogenbosch

\vspace{1em}

\noindent \textbf{Academic Supervisor:}\\
Arthur Kluitmans\\
atjm.kluitmans@avans.nl

\vspace{1em}

\noindent \textbf{Independent Examiner:}\\
Unknown at the moment

\vspace{1em}

\noindent \textbf{Executing Party:}\\
Seger Sars\\
seger.sars@topic.nl\\
Student number: 2184122\\
\\
and\\
\\
Wouter Boerenkamps\\
wouter.boerenkamps@topic.nl\\
Student number: 2171721

\newpage

\section*{Version Control}

\begin{tabular}{|c|l|c|}
    \hline
    Version & Description & Date \\
    \hline
    1.2 & Adjustments in line with new graduation guide & 21-11-2021 \\
    1.3 & Adjusted phasing chapter & 26-02-2024 \\
    \hline
\end{tabular}

\newpage

\tableofcontents

\newpage
%wouter
\section{Background}

\subsection{Organization}
TOPIC Embedded Systems is a company with a no-nonsense attitude and a strong
family-driven culture. Skilled technical consultants provide expertise in
embedded software and digital hardware. Services include executing complete
development projects either on-site or under in-house management from the office
in Best.

TOPIC Healthcare Solutions focuses on clinical workflow optimization using
advanced digital technologies. With over 25 years of experience, the embedded
market and its challenges are well understood. A team of more than 100 engineers
stands ready to support innovative product development with high-quality
technical solutions.

Our daily expertise, embedded in your future.
\subsection{Project}\label{project-background}
The Stewart platform development is a umbrella project to experiment with
different embedded software and FPGA firmware implementation strategies. The aim
is to test alternative implementation approaches, finding the balance between
implementation choice, design complexity, integration effort and cost. The
development of the Stewart platform is driven by a series of graduation projects,
increasing the (software) complexity step-by-step. The history of the project
can be summarized as: 
\begin{itemize} 
\item[] \textbf{Mitchell Broeren} Was
responsible for creating the physical platform implementation: the mechanical
aspects as well as the first driver implementation. He had to overcome all the
aspects involved building a design from scratch. Delivered a working platform
with some needs for improvement.
\item[]\textbf{Chiel van de
Camp} Improved the motor drive quality and processing model. Also introduced the
touch-screen balance board as pressure sensor. Demonstrated that proper
balancing act with the board and a ball was reliable possible. And also
demonstrated that the pressure sensor may not be the most optimal choice.
\item[]\textbf{Jesper Weijnen} Changed the inverse kinetic model implementation
of the Stewart platform into a look-up table in the context of improving on
latency of the control flow using the microcontroller. The idea of the balancing
board was replaced by an object tracking implementation, where the platform
should keep the object to track in the middle of the screen. Object recognition
was introduced using a USB webcam on a PC\@. The experiment clearly illustrated
that real-time video processing performance requires dedicated processing
hardware, unless serious compromises are made or very application specific
optimizations are considered. An experiment to get the video processing
implemented in the FPGA using high-level-synthesis (C-code synthesis straight
into FPGA logic) was also successful but could not be tested live due to a
limitation in time. 
\item[]\textbf{Stef van Stipdonk} Has created a specific
skeleton VWC BSP to have multiple development kits collaborate. The user
interface to interact with the devices runs over a webserver. Both functions
shall be re-used when extending the demonstrator platform. \end{itemize}
\subsection{Parties}
This project is being developed mostly by different interns combined with the input of the intern counselor and supporting employees.


\newpage
%wouter
\section{Project Result}


\subsection{Problem Analysis} As described in section \ref{project-background},
the current version of the project is a functional Stewart platform that uses a
touchpad sensor to determine the ball's position. The system is controlled by an
STM32 microcontroller, which utilizes an inverse kinematics lookup table to
operate the motors via motor drivers.

However, the touchpad sensor introduces significant latency, which negatively
impacts the system’s responsiveness and performance. This limitation affects the
accuracy and real-time capabilities of the platform. Topic has researched the
possibility of replacing the touchpad with a camera-based machine vision system
to reduce latency and improve tracking precision.

Additionally, there is a need to migrate the current demonstration setup from
the STM32 microcontroller to the Miami Plus ZU9 board.
the project will explore communication options between the CPU and the FPGA,
particularly through the built-in DMA module, to optimize data transfer.

\subsection{Problem Definition} To enhance the current setup, this project will
integrate a previously selected camera to detect the ball's position, speed, and
direction, improving platform control. Additionally, the existing STM32-based
implementation is migrated to the Miami Plus ZU9 board. In this new
configuration, the FPGA will serve as the primary controller for machine vision
processing and as the interface between the board and the motor controllers.

These functionalities must be integrated and managed by a custom
Linux distribution, which will combine the various components and enable
demonstration management via a web interface.

\subsection{Objective}
The Objective of this project is to gain more knoweldge about machine vision and motor control using the FPGA 
and to improve the current setup to reduce latency and improve overal performance.
\subsection{Final Result}


\newpage
%seger
\section{Phasing}
It is important to have structure within a project. Therefore, a project is often divided into different phases. This chapter further explains the phases of your project and the activities that will be undertaken in each phase.

Examples of possible phasing structures:
\begin{itemize}
    \item Analysis, design, implementation, testing.
    \item Analysis, functional design, technical design, implementation, verification, validation.
    \item Initiation, definition, design, preparation, implementation, maintenance.
\end{itemize}
Other phasing structures are also possible. Choose the best fit for your project.

\subsection{Phase Descriptions}
Each phase will involve certain activities and yield specific results. The planning and task distribution of these activities is further explained in the \textbf{Planning} chapter. Some phases may overlap.

\subsubsection{Phase 1}
[Brief description]
\begin{itemize}
    \item Activities
    \begin{itemize}
        \item Activity 1
        \item Activity 2
    \end{itemize}
    \item Results
    \begin{itemize}
        \item Result 1
        \item Result 2
    \end{itemize}
\end{itemize}


\newpage
%seger
\section{Quality}
\subsection{Product Quality}
\subsection{Documentation Quality}


\newpage
%wouter
\section{Confidentiality}


\newpage
%wouter
\section{Project Organization}
\subsection{Organization}
\subsection{Roles and Personnel}
\subsection{Information}


\newpage
%seger
\section{Planning}


\newpage
%wouter
\section{Project Boundaries}


\newpage
%wouter
\section{Costs and Benefits}


\newpage
%seger
\section{Risks}
\subsection{Technical Risks}
\subsection{Project Management Risks}
\subsection{Measures}
This section describes the measures taken to prevent the above risks or the actions to be taken if they occur.


\newpage

\section{List of Figures}

\section{List of Tables}

\section{Appendices}

\end{document}
