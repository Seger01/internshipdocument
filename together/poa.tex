\documentclass{article}
\usepackage{geometry}
\usepackage{hyperref}
\usepackage{enumitem}

\geometry{a4paper, margin=1in}

\title{Plan of Attack}
\author{Seger Sars and Wouter Boerenkamps}
\date{04-02-2025}

\begin{document}

\maketitle

\newpage

\noindent \textbf{Client:}\\
Topic Embedded Systems\\
Engineering department\\
Materiaalweg 4, 5681 RJ Best\\
+31 499336979\\

\vspace{1em}

\noindent \textbf{Company Supervisor:}\\
Dirk van den Heuvel\\
dirk.van.den.heuvel@topic.nl\\
Position: Product Manager and Principal Consultant at TOPIC Embedded Systems\\
\\
and\\
\\
Britta Claes\\
britta.claes@topic.nl\\
Position: Business Manager, People Manager, and Account Manager

\vspace{1em}

\noindent \textbf{Educational Institution:}\\
Avans University of Applied Sciences\\
Academy for Industry and Informatics\\
Onderwijsboulevard 215\\
5223 DE 's-Hertogenbosch

\vspace{1em}

\noindent \textbf{Academic Supervisor:}\\
Arthur Kluitmans\\
atjm.kluitmans@avans.nl

\vspace{1em}

\noindent \textbf{Independent Examiner:}\\
Unknown at the moment

\vspace{1em}

\noindent \textbf{Executing Party:}\\
Seger Sars\\
seger.sars@topic.nl\\
Student number: 2184122\\
\\
and\\
\\
Wouter Boerenkamps\\
wouter.boerenkamps@topic.nl\\
Student number: 2171721

\newpage

\section*{Version Control}

\begin{tabular}{|c|l|c|}
    \hline
    Version & Description & Date \\
    \hline
    1.2 & Adjustments in line with new graduation guide & 21-11-2021 \\
    1.3 & Adjusted phasing chapter & 26-02-2024 \\
    \hline
\end{tabular}

\newpage

\tableofcontents

\newpage
%wouter
\section{Background}
\subsection{Organization}
\subsection{Project}
\subsection{Parties}


\newpage
%wouter
\section{Project Result}
\subsection{Problem Analysis}
\subsection{Problem Definition}
\subsection{Objective}
\subsection{Final Result}


\newpage
%seger
\section{Phasing}
It is important to have structure within a project. Therefore, a project is often divided into different phases. This chapter further explains the phases of your project and the activities that will be undertaken in each phase.

Examples of possible phasing structures:
\begin{itemize}
    \item Analysis, design, implementation, testing.
    \item Analysis, functional design, technical design, implementation, verification, validation.
    \item Initiation, definition, design, preparation, implementation, maintenance.
\end{itemize}
Other phasing structures are also possible. Choose the best fit for your project.

\subsection{Phase Descriptions}
Each phase will involve certain activities and yield specific results. The planning and task distribution of these activities is further explained in the \textbf{Planning} chapter. Some phases may overlap.

\subsubsection{Phase 1}
[Brief description]
\begin{itemize}
    \item Activities
    \begin{itemize}
        \item Activity 1
        \item Activity 2
    \end{itemize}
    \item Results
    \begin{itemize}
        \item Result 1
        \item Result 2
    \end{itemize}
\end{itemize}


\newpage
%seger
\section{Quality}
\subsection{Product Quality}
\subsection{Documentation Quality}


\newpage
%wouter
\section{Confidentiality}


\newpage
%wouter
\section{Project Organization}
\subsection{Organization}
\subsection{Roles and Personnel}
\subsection{Information}


\newpage
%seger
\section{Planning}


\newpage
%wouter
\section{Project Boundaries}


\newpage
%wouter
\section{Costs and Benefits}


\newpage
%seger
\section{Risks}
\subsection{Technical Risks}
\subsection{Project Management Risks}
\subsection{Measures}
This section describes the measures taken to prevent the above risks or the actions to be taken if they occur.


\newpage

\section{List of Figures}

\section{List of Tables}

\section{Appendices}

\end{document}
