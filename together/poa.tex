\documentclass{article}
\usepackage{geometry}
\usepackage{hyperref}
\usepackage{enumitem}
\usepackage{booktabs} % For better looking tables
\usepackage{geometry} % To adjust margins
\usepackage{tabularx} % For tables that auto-adjust to the page width
\usepackage{longtable} % For tables across multiple pages

\geometry{a4paper, margin=1in}

\title{Plan of Attack}
\author{Seger Sars and Wouter Boerenkamps}
\date{04-02-2025}

\begin{document}

\maketitle

\newpage

\noindent \textbf{Client:}\\
Topic Embedded Systems\\
Engineering department\\
Materiaalweg 4, 5681 RJ Best\\
+31 499336979\\

\vspace{1em}

\noindent \textbf{Company Supervisor:}\\
Dirk van den Heuvel\\
dirk.van.den.heuvel@topic.nl\\
Position: Product Manager and Principal Consultant at TOPIC Embedded Systems\\
\\
and\\
\\
Britta Claes\\
britta.claes@topic.nl\\
Position: Business Manager, People Manager, and Account Manager

\vspace{1em}

\noindent \textbf{Educational Institution:}\\
Avans University of Applied Sciences\\
Academy for Industry and Informatics\\
Onderwijsboulevard 215\\
5223 DE 's-Hertogenbosch

\vspace{1em}

\noindent \textbf{Academic Supervisor:}\\
Arthur Kluitmans\\
atjm.kluitmans@avans.nl

\vspace{1em}

\noindent \textbf{Independent Examiner:}\\
Unknown at the moment

\vspace{1em}

\noindent \textbf{Executing Party:}\\
Seger Sars\\
seger.sars@topic.nl\\
Student number: 2184122\\
\\
and\\
\\
Wouter Boerenkamps\\
wouter.boerenkamps@topic.nl\\
Student number: 2171721

\newpage

\section*{Version Control}

\begin{tabular}{|c|l|c|}
    \hline
    Version & Description & Date \\
    \hline
    1.2 & Adjustments in line with new graduation guide & 21-11-2021 \\
    1.3 & Adjusted phasing chapter & 26-02-2024 \\
    \hline
\end{tabular}

\newpage

\tableofcontents

\newpage
%wouter
\section{Background}

\subsection{Organization}
TOPIC Embedded Systems is a company with a no-nonsense attitude and a strong
family-driven culture. Skilled technical consultants provide expertise in
embedded software and digital hardware. Services include executing complete
development projects either on-site or under in-house management from the office
in Best.

TOPIC Healthcare Solutions focuses on clinical workflow optimization using
advanced digital technologies. With over 25 years of experience, the embedded
market and its challenges are well understood. A team of more than 100 engineers
stands ready to support innovative product development with high-quality
technical solutions.

Our daily expertise, embedded in your future.
\subsection{Project}\label{project-background}
The Stewart platform development is a umbrella project to experiment with
different embedded software and FPGA firmware implementation strategies. The aim
is to test alternative implementation approaches, finding the balance between
implementation choice, design complexity, integration effort and cost. The
development of the Stewart platform is driven by a series of graduation projects,
increasing the (software) complexity step-by-step. The history of the project
can be summarized as: 
\begin{itemize} 
\item[] \textbf{Mitchell Broeren} Was
responsible for creating the physical platform implementation: the mechanical
aspects as well as the first driver implementation. He had to overcome all the
aspects involved building a design from scratch. Delivered a working platform
with some needs for improvement.
\item[]\textbf{Chiel van de
Camp} Improved the motor drive quality and processing model. Also introduced the
touch-screen balance board as pressure sensor. Demonstrated that proper
balancing act with the board and a ball was reliable possible. And also
demonstrated that the pressure sensor may not be the most optimal choice.
\item[]\textbf{Jesper Weijnen} Changed the inverse kinetic model implementation
of the Stewart platform into a look-up table in the context of improving on
latency of the control flow using the microcontroller. The idea of the balancing
board was replaced by an object tracking implementation, where the platform
should keep the object to track in the middle of the screen. Object recognition
was introduced using a USB webcam on a PC\@. The experiment clearly illustrated
that real-time video processing performance requires dedicated processing
hardware, unless serious compromises are made or very application specific
optimizations are considered. An experiment to get the video processing
implemented in the FPGA using high-level-synthesis (C-code synthesis straight
into FPGA logic) was also successful but could not be tested live due to a
limitation in time. 
\item[]\textbf{Stef van Stipdonk} Has created a specific
skeleton VWC BSP to have multiple development kits collaborate. The user
interface to interact with the devices runs over a webserver. Both functions
shall be re-used when extending the demonstrator platform. \end{itemize}
\subsection{Parties}
This project is being developed mostly by different interns combined with the input of the intern counselor and supporting employees.


\newpage
%wouter
\section{Project Result}


\subsection{Problem Analysis} As described in section \ref{project-background},
the current version of the project is a functional Stewart platform that uses a
touchpad sensor to determine the ball's position. The system is controlled by an
STM32 microcontroller, which utilizes an inverse kinematics lookup table to
operate the motors via motor drivers.

However, the touchpad sensor introduces significant latency, which negatively
impacts the system’s responsiveness and performance. This limitation affects the
accuracy and real-time capabilities of the platform. Topic has researched the
possibility of replacing the touchpad with a camera-based machine vision system
to reduce latency and improve tracking precision.

Additionally, there is a need to migrate the current demonstration setup from
the STM32 microcontroller to the Miami Plus ZU9 board.
the project will explore communication options between the CPU and the FPGA,
particularly through the built-in DMA module, to optimize data transfer.

\subsection{Problem Definition} To enhance the current setup, this project will
integrate a previously selected camera to detect the ball's position, speed, and
direction, improving platform control. Additionally, the existing STM32-based
implementation is migrated to the Miami Plus ZU9 board. In this new
configuration, the FPGA will serve as the primary controller for machine vision
processing and as the interface between the board and the motor controllers.

These functionalities must be integrated and managed by a custom
Linux distribution, which will combine the various components and enable
demonstration management via a web interface.

\subsection{Objective}
The Objective of this project is to gain more knoweldge about machine vision and motor control using the FPGA 
and to improve the current setup to reduce latency and improve overal performance.
\subsection{Final Result}
The final result of this project is a functional stewards platform using the Miami Plus ZU9.
Where the FPGA is used to 

\newpage
%seger
\section{Phasing}
In each phase, a number of activities must be carried out, and each phase will produce results.
The planning and division of tasks for these activities is further explained in the Planning chapter.
It is possible that different phases may overlap.

% \subsection{Plan of Attack}
% This is the initial phase of the project in which the plan for the entire project is formulated. It consists of an overview of the objectives, required resources, planning, and the approach that will be followed.
%
% \begin{itemize}[leftmargin=*, label={}]
%     \item \textbf{Activities:}
%     \begin{itemize}
%         \item Familiarizing oneself with the project.
%         \item Reviewing the work of previous interns and discussing it with the supervisor.
%         \item Formulating the Action Plan.
%     \end{itemize}
%     \item \textbf{Results:}
%     \begin{itemize}
%         \item General knowledge about the project.
%         \item Understanding of previous interns' work and their pitfalls.
%         \item The fully formulated Action Plan.
%     \end{itemize}
% \end{itemize}
\subsection{Plan of Attack}
This is the foundational phase of the project, during which a comprehensive plan is formulated to ensure smooth execution.
This phase includes understanding the project scope, identifying key challenges, defining objectives, and establishing a structured approach.
Effective planning at this stage minimizes risks and enhances project efficiency.

\begin{itemize}[leftmargin=*, label={}]
    \item \textbf{Activities:}
    \begin{itemize}
        \item \textbf{Project Familiarization:}
        \begin{itemize}
            \item Studying project documentation, previous reports, and technical specifications.
            \item Conducting background research on relevant technologies and methodologies.
            \item Understanding the expectations and deliverables from project stakeholders.
        \end{itemize}
        \item \textbf{Engagement with Prior Work:}
        \begin{itemize}
            \item Reviewing the work of previous interns and team members.
            \item Analyzing past implementations, successes, and failures.
            \item Consulting the supervisor and senior members for insights and best practices.
        \end{itemize}
        \item \textbf{Formulating the Action Plan:}
        \begin{itemize}
            \item Defining clear, measurable objectives and milestones.
            \item Identifying dependencies, constraints, and possible risks.
            \item Determining an initial timeline and key deadlines.
            \item Establishing a methodology for project execution.
            \item Setting up tools for task tracking, version control, and documentation.
        \end{itemize}
        \item \textbf{Risk Assessment and Contingency Planning:}
        \begin{itemize}
            \item Identifying potential challenges and roadblocks.
            \item Developing contingency plans for anticipated issues.
            \item Establishing fallback strategies for missed deadlines or unexpected hurdles.
        \end{itemize}
    \end{itemize}
    
    \item \textbf{Results:}
    \begin{itemize}
        \item \textbf{Thorough Understanding of the Project:}
        \begin{itemize}
            \item Clear knowledge of project goals, deliverables, and success criteria.
            \item Familiarity with technical aspects, tools, and methodologies required.
            \item Awareness of Topic standards and best practices.
        \end{itemize}
        %REVIEW: remove
        \item \textbf{Lessons Learned from Previous Work:}
        \begin{itemize}
            \item Understanding past achievements and shortcomings.
            \item Identification of areas for improvement and innovation.
        \end{itemize}
        \item \textbf{Comprehensive Action Plan:}
        \begin{itemize}
            \item Well-defined objectives, timelines, and milestones.
            \item Risk assessment strategies.
            \item A structured workflow and clearly assigned responsibilities.
        \end{itemize}
        \item \textbf{Increased Readiness for Implementation:}
        \begin{itemize}
            \item Clear roadmap ensuring a structured project progression.
            \item Effective coordination mechanisms in place.
        \end{itemize}
    \end{itemize}
\end{itemize}


\subsection{Requirements}
In this phase, the functional and technical requirements of the project are thoroughly defined. This phase ensures that the final product meets all necessary functional expectations and technical specifications. The requirements must be clear, structured, and feasible within the given constraints. A well-defined requirements phase prevents scope creep and ensures alignment between stakeholders.

\begin{itemize}[leftmargin=*, label={}]
    \item \textbf{Activities:}
    \begin{itemize}
        \item \textbf{Understanding the Project Scope:}
        \begin{itemize}
            \item Reviewing initial project documentation and relevant materials.
            \item Identifying the key objectives and expected outcomes.
            \item Understanding any constraints (budget, time, technical limitations).
        \end{itemize}
        \item \textbf{Analyzing the Current System and Architecture:}
        \begin{itemize}
            \item Reviewing the current state and architecture of existing software.
            \item Identifying system dependencies and interactions.
        \end{itemize}
        \item \textbf{Stakeholder Consultation:}
        \begin{itemize}
            \item Discussing project goals and expectations with the supervisor.
        \end{itemize}
        \item \textbf{Defining Functional Requirements:}
        \begin{itemize}
            \item Establishing what the final product must be able to do.
            \item Breaking down features into core functionalities and optional enhancements.
            \item Categorizing requirements using the MoSCoW method (Must-Have, Should-Have, Could-Have, Won't-Have).
        \end{itemize}
        \item \textbf{Defining Technical Requirements:}
        \begin{itemize}
            \item Specifying required programming languages, frameworks, and technologies.
            \item Establishing compatibility requirements with existing infrastructure.
        \end{itemize}
        \item \textbf{Formulating Measurable Requirements:}
        \begin{itemize}
            \item Ensuring requirements meet the SMART (Specific, Measurable, Acceptable, Realistic, and Time-bound) criteria.
        \end{itemize}
        \item \textbf{Documenting the Requirements:}
        \begin{itemize}
            \item Compiling all requirements into a formal requirements specification document.
            \item Conducting reviews and revisions with stakeholders.
        \end{itemize}
    \end{itemize}

    \item \textbf{Results:}
    \begin{itemize}
        \item \textbf{Comprehensive Functional Requirements:}
        \begin{itemize}
            \item Well-defined and categorized functionalities.
            \item Prioritized using the MoSCoW method for structured development.
        \end{itemize}
        \item \textbf{Robust Technical Requirements:}
        \begin{itemize}
            \item Defined system constraints.
            \item Clearly outlined integration points.
            \item Established development tools, frameworks, and infrastructure needs.
        \end{itemize}
        \item \textbf{Requirements Following SMART Criteria}
        \item \textbf{Finalized and Reviewed Requirements Document:}
        \begin{itemize}
            \item Approved and validated by key stakeholders.
            \item Maintained for reference throughout the project lifecycle.
            \item Forms the foundation for future design and development decisions.
        \end{itemize}
    \end{itemize}
\end{itemize}

\subsection{Research Log}
In this phase, research is conducted on various unknown subjects relevant to the project. Any gaps in knowledge, technical uncertainties, or unfamiliar concepts will be systematically explored and documented. The research findings will serve as a foundation for making informed decisions during the development process. All previously unknown aspects will be clearly identified, thoroughly investigated, and recorded in an organized manner.

\begin{itemize}[leftmargin=*, label={}]
    \item \textbf{Activities:}
    \begin{itemize}
        \item \textbf{Identifying Knowledge Gaps:}
        \begin{itemize}
            \item Listing all concepts, technologies, and methodologies that require further understanding.
            \item Defining key research questions to guide the study.
        \end{itemize}
        \item \textbf{Reviewing Existing Work:}
        \begin{itemize}
            \item Analyzing documents and reports from previous interns.
            \item Extracting valuable insights from prior work.
        \end{itemize}
        \item \textbf{Consulting Experts and Supervisors:}
        \begin{itemize}
            \item Engaging in discussions with supervisors and experienced team members.
            \item Asking specific, targeted questions to clarify technical details.
        \end{itemize}
        \item \textbf{Online and Literature Research:}
        \begin{itemize}
            \item Searching the internet for relevant standards and technical papers. 
        \end{itemize}
        \item \textbf{Performing Practical Experiments and Prototyping:}
        \begin{itemize}
            \item Testing small-scale implementations to verify theoretical concepts.
        \end{itemize}
        \item \textbf{Structuring and Documenting Research Findings:}
        \begin{itemize}
            \item Keeping a well-organized research log with detailed notes.
            \item Maintaining a bibliography of all sources consulted for future reference.
        \end{itemize}
    \end{itemize}

    \item \textbf{Results:}
    \begin{itemize}
        \item \textbf{Comprehensive Research Log:}
        \begin{itemize}
            \item A structured document covering most previously unknown topics.
            \item Clearly written explanations and justifications for all findings.
            \item Logical organization for easy reference during project execution.
        \end{itemize}
        \item \textbf{Resolved Knowledge Gaps:}
        \begin{itemize}
            \item All critical uncertainties addressed and understood.
            \item Documented solutions to previously unclear aspects of the project.
        \end{itemize}
        \item \textbf{Validated Theoretical Concepts:}
        \begin{itemize}
            \item Experimental findings supporting or refuting theoretical assumptions.
            \item A well-informed approach for implementation.
        \end{itemize}
        \item \textbf{Enhanced Decision-Making Ability:}
        \begin{itemize}
            \item Clear guidelines on best practices and recommended approaches.
            \item Improved confidence in choosing appropriate tools and techniques.
            \item Reduced risk of unforeseen challenges due to incomplete knowledge.
        \end{itemize}
    \end{itemize}
\end{itemize}


\subsection{Design}
In this phase, the project's design is developed based on the gathered requirements.
The design process ensures that the system architecture, user interface, and other essential components are well-structured and optimized for maintainability.
The goal is to create a clear blueprint that guides implementation while minimizing risks and uncertainties.

\begin{itemize}[leftmargin=*, label={}]
    \item \textbf{Activities:}
    \begin{itemize}
        \item \textbf{Understanding Stakeholder Expectations:}
        \begin{itemize}
            \item Interviewing the supervisor to understand their architectural vision.
            \item Gathering feedback from stakeholders on usability and functionality needs.
            \item Analyzing potential trade-offs between different design choices.
        \end{itemize}
        \item \textbf{Defining System Architecture:}
        \begin{itemize}
            \item Identifying core components and their interactions.
            \item Defining APIs, data flows, and communication protocols.
            \item Ensuring scalability, maintainability, and extensibility in design choices.
        \end{itemize}
        \item \textbf{Technology and Tool Selection:}
        \begin{itemize}
            \item Evaluating different programming languages, frameworks, and libraries.
            \item Considering performance and compatibility with existing systems.
        \end{itemize}
        \item \textbf{Creating Detailed Design Documents:}
        \begin{itemize}
            \item Developing system diagrams, including:
            \begin{itemize}
                \item High-level architecture diagrams.
                \item Data flow diagrams and sequence diagrams.
            \end{itemize}
            \item Writing detailed design specifications for each system component.
        \end{itemize}
        \item \textbf{Reviewing and Refining the Design:}
        \begin{itemize}
            \item Conducting design reviews with peers and supervisors.
            \item Iterating based on feedback to improve clarity and feasibility.
            \item Ensuring all design decisions are well-documented and justified.
        \end{itemize}
    \end{itemize}

    \item \textbf{Results:}
    \begin{itemize}
        \item \textbf{Comprehensive System Architecture:}
        \begin{itemize}
            \item Clearly defined components and their interactions.
            \item Optimized APIs and communication protocols.
        \end{itemize}
        \item \textbf{Well-Documented Design Specification:}
        \begin{itemize}
            \item A formalized design document containing all architectural decisions.
            \item Visual representations (diagrams, flowcharts) of the system's structure.
            \item Justifications for every major design choice.
        \end{itemize}
        \item \textbf{Approval and Readiness for Implementation:}
        \begin{itemize}
            \item Finalized and validated design document.
            \item Supervisor and stakeholder approval obtained.
            \item Clear roadmap for transitioning to the implementation phase.
        \end{itemize}
    \end{itemize}
\end{itemize}

% \subsection{Product}
%
% This is the phase in which the actual development and construction of the project takes place. Here, the plans are translated into a working product.
%
% \begin{itemize}[leftmargin=*, label={}]
%     \item \textbf{Activities:}
%     \begin{itemize}
%         \item Completing the sprints and finishing all components.
%         \item Resolving unforeseen issues while updating the design and requirements in consultation, if necessary.
%     \end{itemize}
%     \item \textbf{Results:}
%     \begin{itemize}
%         \item At the end of the sprints, a fully functional product that meets all requirements.
%         \item If the project progresses faster than expected, requirements under the "Could" category may be included.
%     \end{itemize}
% \end{itemize}
%
% \subsection{Qualification}
%
% In this phase, the completed product is tested and validated to ensure that it meets the specified requirements and specifications.
%
% \begin{itemize}[leftmargin=*, label={}]
%     \item \textbf{Activities:}
%     \begin{itemize}
%         \item Developing system tests.
%         \item Executing system tests.
%         \item Comparing functionality with the MINT.
%         \item Reporting on test results.
%     \end{itemize}
%     \item \textbf{Results:}
%     \begin{itemize}
%         \item Completed functional tests with documented outcomes.
%     \end{itemize}
% \end{itemize}

\subsection{Product}
This is the phase in which the actual development and construction of the project takes place. Here, the plans are translated into a working product. The development process is iterative, with regular reviews and adjustments to ensure alignment with the project goals and requirements. This phase focuses on delivering a high-quality, functional product that meets stakeholder expectations.

\begin{itemize}[leftmargin=*, label={}]
    \item \textbf{Activities:}
    \begin{itemize}
        \item \textbf{Implementation of Core Components:}
        \begin{itemize}
            \item Developing the main functionalities as outlined in the design specifications.
            \item Writing clean, modular, and well-documented code.
            \item Ensuring adherence to coding standards and best practices.
        \end{itemize}
        \item \textbf{Integration of System Components:}
        \begin{itemize}
            \item Combining individual modules into a cohesive system.
            \item Testing interfaces and interactions between components.
            \item Resolving integration issues and ensuring seamless communication.
        \end{itemize}
        \item \textbf{Iterative Development and Testing:}
        \begin{itemize}
            \item Conducting unit tests and integration tests during development.
            \item Addressing bugs and issues identified during testing.
            \item Refactoring code to improve performance and maintainability.
        \end{itemize}
        \item \textbf{Handling Unforeseen Challenges:}
        \begin{itemize}
            \item Identifying and resolving unexpected technical or design issues.
            \item Updating design and requirements in consultation with stakeholders, if necessary.
            \item Implementing contingency plans to mitigate risks.
        \end{itemize}
        \item \textbf{Including Optional Features:}
        \begin{itemize}
            \item If the project progresses ahead of schedule, implementing "Could-Have" requirements.
            \item Enhancing the product with additional features to improve usability or functionality.
        \end{itemize}
        \item \textbf{Documentation and Version Control:}
        \begin{itemize}
            \item Maintaining detailed documentation of the development process.
            \item Using version control systems to track changes and collaborate effectively.
        \end{itemize}
    \end{itemize}

    \item \textbf{Results:}
    \begin{itemize}
        \item \textbf{Fully Functional Product:}
        \begin{itemize}
            \item A working product that meets all specified requirements.
            \item Core functionalities implemented and tested.
        \end{itemize}
        \item \textbf{Optional Features (if applicable):}
        \begin{itemize}
            \item Additional features included to enhance the product.
        \end{itemize}
        \item \textbf{Comprehensive Documentation:}
        \begin{itemize}
            \item Detailed records of the development process, including code and design changes.
            \item Version control logs for traceability and collaboration.
        \end{itemize}
        \item \textbf{Readiness for Qualification:}
        \begin{itemize}
            \item A stable and tested product ready for validation and testing.
        \end{itemize}
    \end{itemize}
\end{itemize}

\subsection{Qualification}
In this phase, the completed product is tested and validated to ensure that it meets the specified requirements and specifications. The qualification phase is critical for verifying the functionality, performance, and reliability of the product. It ensures that the final deliverable aligns with stakeholder expectations and is ready for deployment.

\begin{itemize}[leftmargin=*, label={}]
    \item \textbf{Activities:}
    \begin{itemize}
        \item \textbf{Developing Test Plans:}
        \begin{itemize}
            \item Creating detailed test cases based on functional and technical requirements.
            \item Defining acceptance criteria for each test case.
            \item Ensuring test coverage for all critical functionalities.
        \end{itemize}
        \item \textbf{Executing System Tests:}
        \begin{itemize}
            \item Conducting functional tests to verify that the product behaves as expected.
            \item Performing performance tests to evaluate system responsiveness and stability.
            \item Running stress tests to identify breaking points and limitations.
        \end{itemize}
        \item \textbf{Comparing Functionality with Requirements:}
        \begin{itemize}
            \item Validating that the product meets all "Must-Have" and "Should-Have" requirements.
            \item Ensuring compliance with technical specifications and constraints.
        \end{itemize}
        \item \textbf{Reporting on Test Results:}
        \begin{itemize}
            \item Documenting test outcomes, including pass/fail status and identified issues.
            \item Providing detailed reports to stakeholders for review and feedback.
        \end{itemize}
        \item \textbf{Addressing Defects and Issues:}
        \begin{itemize}
            \item Fixing bugs and resolving issues identified during testing.
            \item Retesting to confirm that fixes are effective and do not introduce new problems.
        \end{itemize}
        \item \textbf{Final Validation and Approval:}
        \begin{itemize}
            \item Conducting a final review with stakeholders to confirm product readiness.
            \item Obtaining formal approval for deployment or delivery.
        \end{itemize}
    \end{itemize}

    \item \textbf{Results:}
    \begin{itemize}
        \item \textbf{Completed Functional Tests:}
        \begin{itemize}
            \item All test cases executed and documented.
            \item Clear evidence of product functionality and performance.
        \end{itemize}
        \item \textbf{Defect Reports and Resolutions:}
        \begin{itemize}
            \item Detailed records of identified issues and their resolutions.
            \item Confirmation that all critical defects have been addressed.
        \end{itemize}
        \item \textbf{Final Test Report:}
        \begin{itemize}
            \item A comprehensive report summarizing test activities and outcomes.
            \item Recommendations for further improvements or enhancements.
        \end{itemize}
        \item \textbf{Validated and Approved Product:}
        \begin{itemize}
            \item A fully tested and validated product ready for deployment.
            \item Stakeholder approval confirming that the product meets all requirements.
        \end{itemize}
    \end{itemize}
\end{itemize}

\newpage
%seger
\section{Quality}
\subsection{Product Quality}
\subsection{Documentation Quality}


\newpage
%wouter
\section{Confidentiality}


\newpage
%wouter
\section{Project Organization}
\subsection{Organization}
\subsection{Roles and Personnel}
\subsection{Information}


\newpage
%seger
\section{Planning}
Deadlines in \textbf{bold} are mandated by school, all other deadlines are set by the team (and therefore theoretically flexible).
Unless a specific day and/or time is specified, all deadlines are on friday, 17:00.
\begin{longtable}{|l|p{0.4\textwidth}|p{0.4\textwidth}|}
    \hline
    \textbf{Week number} & \textbf{Activity}                         & \textbf{Deadlines}                      \\ \hline
    \endfirsthead

    \hline
    \textbf{Week number} & \textbf{Activity}                         & \textbf{Deadlines}                      \\ \hline
    \endhead

    \hline \multicolumn{3}{r}{\textit{Continued on the next page}}                                             \\ \hline
    \endfoot

    \hline
    \endlastfoot

    1                    & Begin working on plan of attack           & PoA V1                                  \\ \hline
                         & Research potential features               &                                         \\ \hline
    2                    & Research potential features               & Research draft (1st feature per member) \\ \hline
                         & Determine future tasks based off research &                                         \\ \hline
                         & Work on POCs                              &                                         \\ \hline
    3                    & Update PoA according to feedback          & \textbf{PoA (sunday)}                   \\ \hline
                         & Work on POCs                              & Finish first POCs                       \\ \hline
                         & Research potential features               & Research draft (2nd feature per member) \\ \hline
    4                    & Research potential features               & Research V1                             \\ \hline
                         & Start drafting architecture + design      &                                         \\ \hline
                         & Work on POCs                              & Finish second POCs                      \\ \hline
    5                    & Write and implement research feedback     & Research feedback implemented           \\ \hline
                         & Work on POCs                              &                                         \\ \hline
                         & Update architecture + design              &                                         \\ \hline
    6                    & Update architecture + design              & Architecture V1 finished                \\ \hline
                         & Create demos from POCs                    &                                         \\ \hline
                         & Set up automated testing systems          & Automated testing systems set up        \\ \hline
    7                    & Present POCs / demos to class             &                                         \\ \hline
                         & Write test plan                           &                                         \\ \hline
                         & Write design                              &                                         \\ \hline
    8                    & Update PoA                                &                                         \\ \hline
                         & Write design                              & Design V1                               \\ \hline
    9                    & Write engine implementation               &                                         \\ \hline
    10                   & Write engine implementation               &                                         \\ \hline
                         & Review all work first period              & \textbf{All work first period}          \\ \hline
    11                   & Write test plan                           & Test plan finished                      \\ \hline
                         & Write engine implementation               &                                         \\ \hline
    12                   & Write engine implementation               &                                         \\ \hline
                         & Review and update design                  &                                         \\ \hline
                         & Validation app design                     &                                         \\ \hline
    13                   & Write engine implementation               & End of week Engine V1 (all test cases passed)       \\ \hline
                         & Write test plan                           & \textbf{Present testplan}               \\ \hline
                         & Validation app design                     & Validation app design                   \\ \hline
    14                   & Write engine implementation               &                                         \\ \hline
                         & Write validation app                      &                                         \\ \hline
    15                   & Write engine implementation               &                                         \\ \hline
                         & Write validation app                      &                                         \\ \hline
    16                   & Optimize / refine engine                  &                                         \\ \hline
                         & Write validation app                      & Finish validation app                   \\ \hline
    17                   & Optimize / refine engine                  & Engine finalized                        \\ \hline
                         &                                           & Showcase engine and validation app      \\ \hline
    18                   &                                           &                                         \\ \hline
    19                   &                                           &                                         \\ \hline
    20                   &                                           &                                         \\ \hline
\end{longtable}
\newpage

\newpage
%wouter
\section{Project Boundaries}


\newpage
%wouter
\section{Costs and Benefits}


\newpage
%seger
\section{Risks}
\subsection{Technical Risks}
\subsection{Project Management Risks}
\subsection{Measures}
This section describes the measures taken to prevent the above risks or the actions to be taken if they occur.


\newpage

\section{List of Figures}

\section{List of Tables}

\section{Appendices}

\end{document}
