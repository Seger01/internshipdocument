\documentclass{article}
\usepackage{geometry}
\usepackage{hyperref}
\usepackage{enumitem}
\usepackage{booktabs} % For better looking tables
\usepackage{geometry} % To adjust margins
\usepackage{tabularx} % For tables that auto-adjust to the page width
\usepackage{longtable} % For tables across multiple pages
\usepackage[backend=biber, style=alphabetic, sorting=ynt]{biblatex} 
\addbibresource{sample.bib}
\geometry{a4paper, margin=1in}

\title{Plan of Attack}
\author{Seger Sars and Wouter Boerenkamps}
\date{04-02-2025}

\begin{document}

\maketitle

\newpage

\noindent \textbf{Client:}\\
Topic Embedded Systems\\
Engineering department\\
Materiaalweg 4, 5681 RJ Best\\
+31 499336979\\

\vspace{1em}

\noindent \textbf{Company Supervisor:}\\
Dirk van den Heuvel\\
dirk.van.den.heuvel@topic.nl\\
Position: Product Manager and Principal Consultant at TOPIC Embedded Systems\\
\\
and\\
\\
Britta Claes\\
britta.claes@topic.nl\\
Position: Business Manager, People Manager, and Account Manager

\vspace{1em}

\noindent \textbf{Educational Institution:}\\
Avans University of Applied Sciences\\
Academy for Industry and Informatics\\
Onderwijsboulevard 215\\
5223 DE 's-Hertogenbosch

\vspace{1em}

\noindent \textbf{Academic Supervisor:}\\
Arthur Kluitmans\\
atjm.kluitmans@avans.nl

\vspace{1em}

\noindent \textbf{Independent Examiner:}\\
Unknown at the moment

\vspace{1em}

\noindent \textbf{Executing Party:}\\
Seger Sars\\
seger.sars@topic.nl\\
Student number: 2184122\\
\\
and\\
\\
Wouter Boerenkamps\\
wouter.boerenkamps@topic.nl\\
Student number: 2171721

\newpage

\section*{Version Control}

\begin{tabular}{|c|l|c|}
    \hline
    Version & Description & Date \\
    \hline
    0.1 & Setup initial document & 04-02-2025 \\
    \hline
\end{tabular}

\newpage

\tableofcontents

\newpage

\section{Project Background (Wouter)}


\subsection{Organization}
% Review: NOOOOOO I DON'T WANT TO READ THE SALES PITCH
\textbf{Topic Embedded Systems}, founded in 1996, is a Dutch company
specializing in embedded system development. With over 25 years of experience,
the company offers a variety of services, including consultancy, project
execution, and the development of System-On-Modules (SoMs) to accelerate product
development.

\paragraph{Services Offered} 
\begin{itemize} 
    \item \textbf{Consultancy:} Topic
        provides expert advice and remote consultancy to enhance clients' development
        teams. 
    \item \textbf{Farm Out:} Engineers from Topic are embedded directly into
        clients' development processes, offering hands-on support. 
    \item \textbf{Projects:} Topic manages in-house projects from design and development
        to production and market introduction. 
    \item \textbf{Products:} They develop and
        produce high-quality System-On-Modules that are scalable in performance and
functionality. \end{itemize}

\paragraph{Expertise Areas} 
\begin{itemize} 
    \item \textbf{Software Development:}
        Topic engineers excel in embedded software, web application software, and cloud
        services. 
    \item \textbf{Board Development:} Specialization in electronic circuit
        development and Printed Circuit Board (PCB) production. 
    \item \textbf{FPGA Development:} Topic is a Xilinx Premium Partner, offering extensive knowledge in
        FPGA development. 
    \item \textbf{Math Modelling AI:} Expertise in math modeling
        facilitates the implementation of algorithms on embedded systems and accelerates
        such algorithms on embedded devices. 
\end{itemize}

\subsection{Project}
\label{project-background}

%Review: This shouls not include the aim of the project just the background and history

The Stewart platform development is a umbrella project to experiment with
different embedded software and FPGA firmware implementation strategies. This project was started to 
test alternative implementation approaches, finding the balance between
implementation choice, design complexity, integration effort and cost. The
development of the Stewart platform is driven by a series of graduation projects,
increasing the (software) complexity step-by-step. The history of the project
can be summarized as: 
\begin{itemize} 
    \item[] \textbf{Mitchell Broeren} Was
        responsible for creating the physical platform implementation: the mechanical
        aspects as well as the first driver implementation. He had to overcome all the
        aspects involved building a design from scratch. Delivered a working platform
        with some needs for improvement.
    \item[]\textbf{Chiel van de
        Camp} Improved the motor drive quality and processing model. Also introduced the
        touch-screen balance board as pressure sensor. Demonstrated that proper
        balancing act with the board and a ball was reliable possible. And also
        demonstrated that the pressure sensor may not be the most optimal choice.
    \item[]\textbf{Jesper Weijnen} Changed the inverse kinetic model implementation
        of the Stewart platform into a look-up table in the context of improving on
        latency of the control flow using the microcontroller. The idea of the balancing
        board was replaced by an object tracking implementation, where the platform
        should keep the object to track in the middle of the screen. Object recognition
        was introduced using a USB webcam on a PC\@. The experiment clearly illustrated
        that real-time video processing performance requires dedicated processing
        hardware, unless serious compromises are made or very application specific
        optimizations are considered. An experiment to get the video processing
        implemented in the FPGA using high-level-synthesis (C-code synthesis straight
        into FPGA logic) was also successful but could not be tested live due to a
        limitation in time. 
    \item[]\textbf{Stef van Stipdonk} Has created a specific
        skeleton VWC BSP to have multiple development kits collaborate. The user
        interface to interact with the devices runs over a webserver. Both functions
shall be re-used when extending the demonstrator platform. \end{itemize}
\subsection{Parties}

This project is developed and executed as a graduation internship assignment.
The two primary parties involved in this project are:

\begin{itemize} 
    \item \textbf{Avans Hogeschool:} The educational institution
        providing the internship program. Avans Hogeschool is responsible for overseeing
        the interns' progress, providing educational support, and consulting on academic
        matters. Additionally, an educational consultant from Avans Hogeschool is
        involved to ensure that the internship aligns with the educational requirements
        and the students' learning outcomes. 
    \item \textbf{Topic Embedded Systems:} The
        company providing the internship assignment and the consultancy during the
        project. Topic Embedded Systems is responsible for offering the project topic,
        guidance, and technical consultancy. The company also evaluates the interns'
        progress, providing real-world experience in embedded systems development and
        professional development. 
\end{itemize}




\newpage
%TODO: goal should be different for both duo partners
%TODO: finish this section
\section{Project Result (Wouter)}


\subsection{Problem Analysis} As described in section \ref{project-background},
the current version of the project is a functional Stewart platform that uses a
touchpad sensor to determine the ball's position. The system is controlled by an
STM32 microcontroller, which utilizes an inverse kinematics lookup table to
operate the motors via motor drivers.

However, the touchpad sensor introduces significant latency, which negatively
impacts the system’s responsiveness and performance. This limitation affects the
accuracy and real-time capabilities of the platform. Topic has researched the
possibility of replacing the touchpad with a camera-based machine vision system
to reduce latency and improve tracking precision.

Additionally, there is a need to migrate the current demonstration setup from
the STM32 microcontroller to the Miami Plus ZU9 board.
the project will explore communication options between the CPU and the FPGA,
particularly through the built-in DMA module, to optimize data transfer.

\subsection{Problem Definition} 
%Review: formal english so no contractions like ball's and the entire sentence it is in does not read well because of the large amount of comma's
%Review2: This entire subsection does not really include a problem definition, it is more of a continuation of the problem analysis or project ojective

To enhance the current setup, this project will
integrate a previously selected camera to detect the ball's position, speed, and
direction, improving platform control. Additionally, the existing STM32-based
implementation is migrated to the Miami Plus ZU9 board. In this new
configuration, the FPGA will serve as the primary controller for machine vision
processing and as the interface between the board and the motor controllers.

These functionalities must be integrated and managed by a custom
Linux distribution, which will combine the various components and enable
demonstration management via a web interface.

\subsection{Objective}
The Objective of this project is to gain more knowledge about machine vision and motor control using the FPGA 
and to improve the current setup to reduce latency and improve overall performance.
\subsection{Final Result}
The final result of this project is a functional stewards platform using the Miami Plus ZU9.
Where the FPGA is used to 
%Review: quite empty

\newpage
\section{Phasing (Seger)}
In each phase, a number of activities must be carried out, and each phase will produce results.
The planning and division of tasks for these activities is further explained in the Planning chapter.
It is possible that different phases may overlap.

\subsection{Plan of Attack}
This is the foundational phase of the project, during which a comprehensive plan is formulated to ensure smooth execution.
This phase includes understanding the project scope, identifying key challenges, defining objectives, and establishing a structured approach.
Effective planning at this stage minimizes risks and enhances project efficiency.

\begin{itemize}[leftmargin=*, label={}]
    \item \textbf{Activities:}
        \begin{itemize}
            \item \textbf{Project Familiarization:}
                \begin{itemize}
                    \item Studying project documentation, previous reports, and technical specifications.
                    \item Conducting background research on relevant technologies and methodologies.
                    \item Understanding the expectations and deliverables from project stakeholders.
                \end{itemize}
            \item \textbf{Engagement with Prior Work:}
                \begin{itemize}
                    \item Reviewing the work of previous interns and team members.
                    \item Analyzing past implementations, successes, and failures.
                    \item Consulting the supervisor and senior members for insights and best practices.
                \end{itemize}
            \item \textbf{Formulating the Action Plan:}
                \begin{itemize}
                    \item Defining clear, measurable objectives and milestones.
                    \item Identifying dependencies, constraints, and possible risks.
                    \item Determining an initial timeline and key deadlines.
                    \item Establishing a methodology for project execution.
                    \item Setting up tools for task tracking, version control, and documentation.
                \end{itemize}
            \item \textbf{Risk Assessment and Contingency Planning:}
                \begin{itemize}
                    \item Identifying potential challenges and roadblocks.
                    \item Developing contingency plans for anticipated issues.
                    \item Establishing fallback strategies for missed deadlines or unexpected hurdles.
                \end{itemize}
        \end{itemize}

    \item \textbf{Results:}
        \begin{itemize}
            \item \textbf{Thorough Understanding of the Project:}
                \begin{itemize}
                    \item Clear knowledge of project goals, deliverables, and success criteria.
                    \item Familiarity with technical aspects, tools, and methodologies required.
                    \item Awareness of Topic standards and best practices.
                \end{itemize}
            \item \textbf{Comprehensive Action Plan:}
                \begin{itemize}
                    \item Well-defined objectives, timelines, and milestones.
                    \item Risk assessment strategies.
                    \item A structured workflow and clearly assigned responsibilities.
                \end{itemize}
            \item \textbf{Increased Readiness for Implementation:}
                \begin{itemize}
                    \item Clear roadmap ensuring a structured project progression.
                    \item Effective coordination mechanisms in place.
                \end{itemize}
        \end{itemize}
\end{itemize}


\subsection{Requirements}
In this phase, the functional and technical requirements of the project are thoroughly defined. This phase ensures that the final product meets all necessary functional expectations and technical specifications. The requirements must be clear, structured, and feasible within the given constraints. A well-defined requirements phase prevents scope creep and ensures alignment between stakeholders.

\begin{itemize}[leftmargin=*, label={}]
    \item \textbf{Activities:}
        \begin{itemize}
            \item \textbf{Understanding the Project Scope:}
                \begin{itemize}
                    \item Reviewing initial project documentation and relevant materials.
                    \item Identifying the key objectives and expected outcomes.
                    \item Understanding any constraints (budget, time, technical limitations).
                \end{itemize}
            \item \textbf{Analyzing the Current System and Architecture:}
                \begin{itemize}
                    \item Reviewing the current state and architecture of existing software.
                    \item Identifying system dependencies and interactions.
                \end{itemize}
            \item \textbf{Defining Functional Requirements:}
                \begin{itemize}
                    \item Establishing what the final product must be able to do.
                    \item Breaking down features into core functionalities and optional enhancements.
                    \item Categorizing requirements using the MoSCoW method (Must-Have, Should-Have, Could-Have, Won't-Have).
                \end{itemize}
            \item \textbf{Defining Technical Requirements:}
                \begin{itemize}
                    \item Specifying required programming languages, frameworks, and technologies.
                    \item Establishing compatibility requirements with existing infrastructure.
                \end{itemize}
            \item \textbf{Formulating Measurable Requirements:}
                \begin{itemize}
                    \item Ensuring requirements meet the SMART (Specific, Measurable, Acceptable, Realistic, and Time-bound) criteria.
                \end{itemize}
            \item \textbf{Documenting the Requirements:}
                \begin{itemize}
                    \item Compiling all requirements into a formal requirements specification document.
                    \item Conducting reviews and revisions with stakeholders.
                \end{itemize}
        \end{itemize}

    \item \textbf{Results:}
        \begin{itemize}
            \item \textbf{Comprehensive Functional Requirements:}
                \begin{itemize}
                    \item Well-defined and categorized functionalities.
                    \item Prioritized using the MoSCoW method for structured development.
                \end{itemize}
            \item \textbf{Robust Technical Requirements:}
                \begin{itemize}
                    \item Defined system constraints.
                    \item Clearly outlined integration points.
                    \item Established development tools, frameworks, and infrastructure needs.
                \end{itemize}
            \item \textbf{Requirements Following SMART Criteria}
            \item \textbf{Finalized and Reviewed Requirements Document:}
                \begin{itemize}
                    \item Approved and validated by key stakeholders.
                    \item Maintained for reference throughout the project lifecycle.
                    \item Forms the foundation for future design and development decisions.
                \end{itemize}
        \end{itemize}
\end{itemize}

\subsection{Research Log}
In this phase, research is conducted on various unknown subjects relevant to the project. Any gaps in knowledge, technical uncertainties, or unfamiliar concepts will be systematically explored and documented. The research findings will serve as a foundation for making informed decisions during the development process. All previously unknown aspects will be clearly identified, thoroughly investigated, and recorded in an organized manner.

\begin{itemize}[leftmargin=*, label={}]
    \item \textbf{Activities:}
        \begin{itemize}
            \item \textbf{Identifying Knowledge Gaps:}
                \begin{itemize}
                    \item Listing all concepts, technologies, and methodologies that require further understanding.
                    \item Defining key research questions to guide the study.
                \end{itemize}
            \item \textbf{Reviewing Existing Work:}
                \begin{itemize}
                    \item Analyzing documents and reports from previous interns.
                    \item Extracting valuable insights from prior work.
                \end{itemize}
            \item \textbf{Consulting Experts and Supervisors:}
                \begin{itemize}
                    \item Engaging in discussions with supervisors and experienced team members.
                    \item Asking specific, targeted questions to clarify technical details.
                \end{itemize}
            \item \textbf{Online and Literature Research:}
                \begin{itemize}
                    \item Searching the internet for relevant standards and technical papers. 
                \end{itemize}
            \item \textbf{Performing Practical Experiments and Prototyping:}
                \begin{itemize}
                    \item Testing small-scale implementations to verify theoretical concepts.
                \end{itemize}
            \item \textbf{Structuring and Documenting Research Findings:}
                \begin{itemize}
                    \item Keeping a well-organized research log with detailed notes.
                    \item Maintaining a bibliography of all sources consulted for future reference.
                \end{itemize}
        \end{itemize}

    \item \textbf{Results:}
        \begin{itemize}
            \item \textbf{Comprehensive Research Log:}
                \begin{itemize}
                    \item A structured document covering most previously unknown topics.
                    \item Clearly written explanations and justifications for all findings.
                    \item Logical organization for easy reference during project execution.
                \end{itemize}
            \item \textbf{Resolved Knowledge Gaps:}
                \begin{itemize}
                    \item All critical uncertainties addressed and understood.
                    \item Documented solutions to previously unclear aspects of the project.
                \end{itemize}
            \item \textbf{Validated Theoretical Concepts:}
                \begin{itemize}
                    \item Experimental findings supporting or refuting theoretical assumptions.
                    \item A well-informed approach for implementation.
                \end{itemize}
            \item \textbf{Enhanced Decision-Making Ability:}
                \begin{itemize}
                    \item Clear guidelines on best practices and recommended approaches.
                    \item Improved confidence in choosing appropriate tools and techniques.
                    \item Reduced risk of unforeseen challenges due to incomplete knowledge.
                \end{itemize}
        \end{itemize}
\end{itemize}


\subsection{Design}
In this phase, the project's design is developed based on the gathered requirements.
The design process ensures that the system architecture, user interface, and other essential components are well-structured and optimized for maintainability.
The goal is to create a clear blueprint that guides implementation while minimizing risks and uncertainties.

\begin{itemize}[leftmargin=*, label={}]
    \item \textbf{Activities:}
        \begin{itemize}
            \item \textbf{Understanding Stakeholder Expectations:}
                \begin{itemize}
                    \item Interviewing the supervisor to understand their architectural vision.
                    \item Gathering feedback from stakeholders on usability and functionality needs.
                    \item Analyzing potential trade-offs between different design choices.
                \end{itemize}
            \item \textbf{Defining System Architecture:}
                \begin{itemize}
                    \item Identifying core components and their interactions.
                    \item Defining APIs, data flows, and communication protocols.
                    \item Ensuring scalability, maintainability, and extensibility in design choices.
                \end{itemize}
            \item \textbf{Technology and Tool Selection:}
                \begin{itemize}
                    \item Evaluating different programming languages, frameworks, and libraries.
                    \item Considering performance and compatibility with existing systems.
                \end{itemize}
            \item \textbf{Creating Detailed Design Documents:}
                \begin{itemize}
                    \item Developing system diagrams, including:
                        \begin{itemize}
                            \item High-level architecture diagrams.
                            \item Data flow diagrams and sequence diagrams.
                        \end{itemize}
                    \item Writing detailed design specifications for each system component.
                \end{itemize}
            \item \textbf{Reviewing and Refining the Design:}
                \begin{itemize}
                    \item Conducting design reviews with peers and supervisors.
                    \item Iterating based on feedback to improve clarity and feasibility.
                    \item Ensuring all design decisions are well-documented and justified.
                \end{itemize}
        \end{itemize}

    \item \textbf{Results:}
        \begin{itemize}
            \item \textbf{Comprehensive System Architecture:}
                \begin{itemize}
                    \item Clearly defined components and their interactions.
                    \item Optimized APIs and communication protocols.
                \end{itemize}
            \item \textbf{Well-Documented Design Specification:}
                \begin{itemize}
                    \item A formalized design document containing all architectural decisions.
                    \item Visual representations (diagrams, flowcharts) of the system's structure.
                    \item Justifications for every major design choice.
                \end{itemize}
            \item \textbf{Approval and Readiness for Implementation:}
                \begin{itemize}
                    \item Finalized and validated design document.
                    \item Supervisor and stakeholder approval obtained.
                    \item Clear roadmap for transitioning to the implementation phase.
                \end{itemize}
        \end{itemize}
\end{itemize}

\subsection{Product}
This is the phase in which the actual development and construction of the project takes place. Here, the plans are translated into a working product. The development process is iterative, with regular reviews and adjustments to ensure alignment with the project goals and requirements. This phase focuses on delivering a high-quality, functional product that meets stakeholder expectations.

\begin{itemize}[leftmargin=*, label={}]
    \item \textbf{Activities:}
        \begin{itemize}
            \item \textbf{Implementation of Core Components:}
                \begin{itemize}
                    \item Developing the main functionalities as outlined in the design specifications.
                    \item Writing clean, modular, and well-documented code.
                \end{itemize}
            \item \textbf{Integration of System Components:}
                \begin{itemize}
                    \item Combining individual modules into a cohesive system.
                    \item Testing interfaces and interactions between components.
                    \item Resolving integration issues and ensuring seamless communication.
                \end{itemize}
            \item \textbf{Iterative Development and Testing:}
                \begin{itemize}
                    \item Conducting unit tests and integration tests during development.
                    \item Refactoring code to improve performance and maintainability.
                \end{itemize}
            \item \textbf{Including Optional Features:}
                \begin{itemize}
                    \item If the project progresses ahead of schedule, implementing "Could-Have" requirements.
                \end{itemize}
            \item \textbf{Documentation and Version Control:}
                \begin{itemize}
                    \item Maintaining detailed documentation of the system during development (architecture and design).
                    \item Using version control systems to track changes.
                \end{itemize}
        \end{itemize}

    \item \textbf{Results:}
        \begin{itemize}
            \item \textbf{Fully Functional Product:}
                \begin{itemize}
                    \item A working product that meets all specified requirements.
                    \item Core functionalities implemented and tested.
                \end{itemize}
            \item \textbf{Optional Features (if applicable):}
                \begin{itemize}
                    \item Additional features included to enhance the product.
                \end{itemize}
            \item \textbf{Comprehensive Documentation:}
                \begin{itemize}
                    \item Detailed records of the development process, including code and design justifications.
                    \item Version control logs for traceability and collaboration.
                \end{itemize}
        \end{itemize}
\end{itemize}

\subsection{Qualification}
In this phase, the completed product is tested and validated to ensure that it meets the specified requirements and specifications. The qualification phase is critical for verifying the functionality, performance, and reliability of the product. It ensures that the final deliverable aligns with stakeholder expectations and is ready for deployment.

\begin{itemize}[leftmargin=*, label={}]
    \item \textbf{Activities:}
        \begin{itemize}
            \item \textbf{Developing Test Plans:}
                \begin{itemize}
                    \item Creating detailed test cases based on functional and technical requirements.
                    \item Defining acceptance criteria for each test case.
                    \item Ensuring test coverage for all critical functionalities.
                \end{itemize}
            \item \textbf{Executing System Tests:}
                \begin{itemize}
                    \item Conducting functional tests to verify that the product behaves as expected.
                    \item Performing performance tests to evaluate system responsiveness and stability.
                    \item Running stress tests to identify breaking points and limitations.
                \end{itemize}
            \item \textbf{Comparing Functionality with Requirements:}
                \begin{itemize}
                    \item Validating that the product meets all "Must-Have" and "Should-Have" requirements.
                    \item Ensuring compliance with technical specifications and constraints.
                \end{itemize}
            \item \textbf{Reporting on Test Results:}
                \begin{itemize}
                    \item Documenting test outcomes, including pass/fail status and identified issues.
                    \item Providing detailed reports to stakeholders for review and feedback.
                \end{itemize}
            \item \textbf{Addressing Defects and Issues:}
                \begin{itemize}
                    \item Fixing bugs and resolving issues identified during testing.
                    \item Retesting to confirm that fixes are effective and do not introduce new problems.
                \end{itemize}
            \item \textbf{Final Validation and Approval:}
                \begin{itemize}
                    \item Conducting a final review with stakeholders to confirm product readiness.
                    \item Obtaining formal approval for deployment or delivery.
                \end{itemize}
        \end{itemize}

    \item \textbf{Results:}
        \begin{itemize}
            \item \textbf{Final Test Report:}
                \begin{itemize}
                    \item All test cases executed and documented.
                    \item Clear evidence of product functionality and performance.
                    \item Recommendations for further improvements or enhancements.
                \end{itemize}
            \item \textbf{Defect Reports and Resolutions:}
                \begin{itemize}
                    \item Detailed records of identified issues and their resolutions.
                \end{itemize}
            \item \textbf{Validated and Approved Product:}
                \begin{itemize}
                    \item A fully tested and validated product.
                    \item Stakeholder approval confirming that the product meets all requirements.
                \end{itemize}
        \end{itemize}
\end{itemize}

\newpage

\section{Quality (Seger)}
The quality of the project is evaluated based on two key aspects: the Product Quality and the Documentation Quality. These aspects ensure that the final deliverables meet the functional, technical, and stakeholder requirements while adhering to best practices in development and documentation.

\subsection{Product Quality}
The product quality is assessed based on its functionality, performance, reliability, and adherence to the specified requirements. The following criteria are used to evaluate the quality of the final product:

\begin{itemize}[leftmargin=*, label={}]
    \item \textbf{Functionality:}
        \begin{itemize}
            \item The product must fully implement all "Must-Have" requirements as defined in the requirements specification.
            \item Optional "Should-Have" and "Could-Have" features should be included if time and resources permit.
            \item The product should perform as expected under normal operating conditions.
        \end{itemize}
    \item \textbf{Performance:}
        \begin{itemize}
            \item The system should meet the performance benchmarks defined during the requirements phase, including latency, responsiveness, and throughput.
            \item Stress tests should confirm that the system can handle specified speeds without failure.
        \end{itemize}
    \item \textbf{Reliability:}
        \begin{itemize}
            \item The product should operate consistently without crashes or unexpected behavior.
            \item All critical bugs identified during testing must be resolved before final delivery.
        \end{itemize}
    \item \textbf{Maintainability:}
        \begin{itemize}
            \item The codebase should be modular, well-documented, and follow best practices for readability and scalability.
            \item The system architecture should allow for easy updates and extensions.
        \end{itemize}
    \item \textbf{User Experience:}
        \begin{itemize}
            \item The user interface (if applicable) should be intuitive and easy to use.
            \item Any user-facing components should provide clear feedback.
        \end{itemize}
\end{itemize}

\subsection{Documentation Quality}
The documentation quality ensures that all aspects of the project are well-documented, enabling future maintenance, knowledge transfer, and stakeholder understanding. The following criteria are used to evaluate the quality of the documentation:

\begin{itemize}[leftmargin=*, label={}]
    \item \textbf{Completeness:}
        \begin{itemize}
            \item All parts of the project (Plan of Attack, Requirements, Research, Design, Product, and Qualification) must be thoroughly documented.
            \item Documentation should include detailed descriptions of activities, results, and decisions made during each phase.
        \end{itemize}
    \item \textbf{Clarity:}
        \begin{itemize}
            \item Documentation should be written in clear, concise language, free of ambiguity.
            \item Technical terms and acronyms should be defined for non-expert readers.
        \end{itemize}
    \item \textbf{Organization:}
        \begin{itemize}
            \item Documentation should be logically structured, with a table of contents, sections, and subsections for easy navigation.
            \item Visual aids such as diagrams, flowcharts, and tables should be used to enhance understanding especially for the design document.
        \end{itemize}
    \item \textbf{Accuracy:}
        \begin{itemize}
            \item All information in the documentation must be accurate and up-to-date.
            \item Any changes to the project scope, design, or implementation should be reflected in the documentation.
        \end{itemize}
    \item \textbf{Version Control:}
        \begin{itemize}
            \item Documentation should be version-controlled, with a clear record of changes and updates.
            \item The latest version of the documentation should be easily accessible to all stakeholders.
        \end{itemize}
    \item \textbf{Compliance with Standards:}
        \begin{itemize}
            \item Documentation should adhere to the standards and templates provided by Topic Embedded Systems and Avans University.
            \item Any additional formatting or style guidelines should be followed consistently.
        \end{itemize}
\end{itemize}

\subsection{Quality Assurance Measures}
To ensure that both product and documentation quality meet the required standards, the following measures will be implemented:
\begin{itemize}[leftmargin=*, label={}]
    \item \textbf{Peer Review:}
        \begin{itemize}
            \item Team members will review each other's work to identify and address issues early in the process.
        \end{itemize}
    \item \textbf{Regular Reviews:}
        \begin{itemize}
            \item Reviews of the product and documentation will be conducted by the team, supervisors, and stakeholders.
            \item Feedback from reviews will be incorporated into the final version of the product and documentation.
        \end{itemize}
    \item \textbf{Coding Standards:}
        \begin{itemize}
            \item Code will be written following best practices and coding standards of Topic.
        \end{itemize}
    \item \textbf{Testing and Validation:}
        \begin{itemize}
            \item Comprehensive testing (unit, integration, system, and acceptance tests) will be performed to validate the product's functionality and performance.
            \item Documentation will be reviewed for accuracy and completeness during the qualification phase.
        \end{itemize}
    \item \textbf{Version Control:}
        \begin{itemize}
            \item All project artifacts, including code, documentation, and design files, will be version-controlled using Git.
            \item Regular commits and branching strategies will be followed to maintain a clean and organized repository.
            \item Documentation will be stored in a shared repository accessible to all team members and stakeholders.
            \item Documentation will be written in Latex for good version control capability and consistency.
        \end{itemize}
\end{itemize}

By adhering to these quality criteria and measures, the project will deliver a high-quality product and documentation that meets the expectations of all stakeholders and ensures long-term maintainability and usability.

\newpage


\section{Confidentiality (Wouter)}

All documentation produced during this project is confidential. This includes
design specifications, research findings, planning documents, and any other
project-related materials. These documents must only be shared with the
project's stakeholders.

Avans University has the right to review and access all project documentation
to ensure compliance with academic and institutional guidelines. However, any
confidential information, especially sensitive or proprietary knowledge, must
not be shared with unauthorized parties.

The rules and regulations outlined in the Non-Disclosure Agreements (NDAs) with
both Avans University and Topic Embedded Systems must be followed. These
agreements specify how confidential information should be handled, stored, and
shared throughout the project.

Confidential documentation must be carefully managed and must not be shared
without proper authorization. Unauthorized distribution or disclosure is
strictly prohibited, unless written consent is provided by the relevant
stakeholders.

\newpage

\section{Project Organization (Wouter)}


\subsection{Organization}
\textbf{Project name}: Steward platform.\\ 
\textbf{Project duration}: 20 weeks (03-02-2025 - 04-07-2025)\\ 
\textbf{Stakeholders}: Topic Embedded Systems

\subsection{Roles and Personnel}
\begin{itemize} 
    \item \textbf{Wouter Boerenkamps} 
\begin{itemize} 
    \item Role: Embedded Software Engineer / intern
    \item Responsibilities: Creating planning, research assignment,
    communication with the educational institution, designing and developing the
    final product. 
    \item Contact: \href{mailto:wrj.boerenkamps@student.avans.nl}{wrj.
boerenkamps@student.avans.nl} 
\end{itemize}

\item \textbf{Arthur Kluitmans}
\begin{itemize} 
    \item Function: Educational Consultant
    \item Responsibilities: educational internship guidance, documentation review, assesing final product and documentation.
    \item Contact: \href{mailto:atjm.kluitmans@avans.nl}{atjm.kluitmans@avans.nl}
\end{itemize}

\item \textbf{Seger Sars}  
\begin{itemize} 
    \item Role: Embedded Software Engineer / intern
    \item Responsibilities: Creating planning, research assignment, communication with the
    educational institution, designing and developing the final product. 
    \item Contact: \href{mailto:sjw.sars@student.avans.nl}{sjw.sars@student.avans.nl}
\end{itemize}

\item \textbf{Dirk van den Heuvel} 
\begin{itemize} 
    \item Role: Product Manager and Principal Consultant at TOPIC Embedded
    Systems
    \item Responsibilities: technical guidance, aproving/discussing design choices
    \item Contact: \href{mailto:dirk.van.
den.heuvel@topic.nl}{dirk.van.den.heuvel@topic.nl} 
\end{itemize}

    \item \textbf{Britta Claes} 
\begin{itemize} 
    \item Role: Business Manager, People Manager, and Account Manager
    \item Responsibilities: general internship guidance
    \item Contact: \href{mailto:britta.claes@topic.
    nl}{britta.claes@topic.nl} 
\end{itemize} 
\end{itemize}
\subsection{Information}
\paragraph{deliverables} \par
This section described the different deliverables that will be created and
submitted during the course of this project. 
\begin{center}
    \begin{tabular}{|l|p{8cm}|} 
        \hline 
        \textbf{Document} & \textbf{Purpose} \\
        \hline 
        Plan of Attack & Defines project goals and approach. \\ 
        \hline
        Requirement document & Specifies technical and functional SMART tasks. \\ 
        \hline
        Research document & Details the research process and findings. \\ 
        \hline 
        Design document & Explains design choices and justifications. \\ 
        \hline 
        Qualification document & Describes test methods and test cases based on requirements. \\
        \hline 
        Research/reflection log & Overview of weekly progress and reflection on important situations \\
        \hline
    \end{tabular} 
\end{center}
\paragraph{Meetings / communication} \par
This paragraph describes the different meetings that take place at fixed intervals throughout the project.
\begin{itemize}
    \item Weekly progress meeting with Dirk van den Heuvel
    \item Weekly research log and reflection
    \item Monthly progress meeting with Britta Claes.
\end{itemize}
\paragraph{Deadlines} 
This paragraph describes the important deadlines set by Avans Hogeschool.
\begin{itemize}
    \item \textbf{Go/No-Go PoA} 17 March 2025 
    \item\textbf{Interim Assessment(Company Supervisor)} 05 May 2025 
    \item\textbf{Final Document Submission} 09 June 2025 
    \item \textbf{Graduation Session} 23 June 2025 
\end{itemize}

\section{Planning (Seger)}
Deadlines in \textbf{bold} are mandated by school, all other deadlines are set by the team (and therefore theoretically flexible).
Unless a specific day and/or time is specified, all deadlines are on friday, 17:00.
\begin{longtable}{|l|p{0.4\textwidth}|p{0.4\textwidth}|}
    \hline
    \textbf{Week} & \textbf{Activity}                         & \textbf{Deadlines}                      \\ \hline
    \endfirsthead

    \hline
    \textbf{Week} & \textbf{Activity}                         & \textbf{Deadlines}                      \\ \hline
    \endhead

    \hline \multicolumn{3}{r}{\textit{Continued on the next page}}                                             \\ \hline
    \endfoot

    \hline
    \endlastfoot

    1 (3-2-2025)  & Get to know the company, work on PoA and plan first meeting with academic supervisor & TBD \\ \hline
    2 (10-2-2025)  & Work on PoA and work on requirements & Submit PoA for review \\ \hline
    3 (17-2-2025)  & Work on requirements, start research & TBD \\ \hline
    4 (24-2-2025)  & Research & Sumbit Requirements document for review \\ \hline
    4a (3-3-2025)  & Research & TBD \\ \hline
    5 (10-3-2025)  & Meeting academic supervisor, research and start Qualification document & TBD \\ \hline
    6 (17-3-2025) & Report on meeting with academic supervisor and research & \textbf{Go/No Go from academic supervisor based on PoA} \\ \hline
    7 (24-3-2025) & Start design and small parts of product & TBD \\ \hline
    8 (31-3-2025) - 9 (7-4-2025) & Work on design and developing small parts of product & TBD \\ \hline
    10 (14-4-2025) & Work on design and developing small parts of product & First version of design done and plan second meeting with academic supervisor \\ \hline
    11 (21-4-2025) & Start fabricating product & TBD \\ \hline
    11a (28-4-2025) & Fabricating product & TBD \\ \hline
    12 (5-5-2025) & Fabricating product and start Qualification & \textbf{Interim assesment from company supervisor} \\ \hline
    13 (12-5-2025) & Fabricating product & TBD \\ \hline
    14 (19-5-2025) & Fabricating product and finishing first version of Qualification, Research and Design & Submit Research, Design, and Qualification for review \\ \hline
    15 (26-5-2025) & Work on finishing product and start testing by Qualification document & TBD \\ \hline
    16 (2-6-2025) & Work on finishing product & TBD \\ \hline
    17 (9-6-2025) & Finish product and do final qualification & \textbf{Final document submission} \\ \hline
    18 (16-6-2025) & TBD & TBD \\ \hline
    19 (23-6-2025) & Create transfer document & \textbf{Graduation session} \\ \hline
    20 (30-6-2025) & Finish transfer document & Finish transfer document \\ \hline
\end{longtable}
\newpage

\newpage
\section{Project Boundaries (Wouter)}
vhdl 2008

\newpage

\section{Costs and Benefits (Wouter)}


\newpage
\section{Risks (Seger)}
\subsection{Technical Risks}
\subsection{Project Management Risks}
\subsection{Measures}
This section describes the measures taken to prevent the above risks or the actions to be taken if they occur.


\newpage

\section{List of Figures}

\section{List of Tables}

\section{Appendices}

\end{document}
