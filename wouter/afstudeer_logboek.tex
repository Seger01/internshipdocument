\documentclass[a4paper,12pt]{article}
\usepackage{longtable}

\title{Afstudeerlogboek}
\author{Wouter Boerenkamps}
\date{2025}

\begin{document}

\maketitle

\newpage

\section*{Week 1}

\subsection*{Voortgang}

\begin{longtable}{|l|l|p{0.09\textwidth}|p{0.7\textwidth}|}
\hline
\textbf{Dag} & \textbf{Datum} & \textbf{Aantal uren} & \textbf{Beschrijving werkzaamheden} \\
\hline
\endfirsthead
\hline
\textbf{Dag} & \textbf{Datum} & \textbf{Aantal uren} & \textbf{Beschrijving werkzaamheden} \\
\hline
\endhead
\hline
\endfoot
\endlastfoot
% Add your data rows below, for example:
Maandag   & 03-02-2025 & 8 & Introductie presentaties en lunch gehad. Laptop gekregen en operating system opgezet.  \\ \hline
Dinsdag   & 04-02-2025 & 8 & De opdracht verduidelijking van Dirk doorgenomen en een groot deel van het plan van aanpak geschreven. \\ \hline
Woensdag  & 05-02-2025 & 8 & Verder gewerkt aan het plan van aanpak en samen met seger vragen opgesteld voor het verduidelijke van de opdracht deze zijn daarna besproken in een gesprek met Dirk.\\ \hline
Donderdag & 06-02-2025 & 0 & Ziek \\ \hline
Vrijdag   & 07-02-2025 & 0 & Ziek \\ \hline

\hline

% Continue adding rows as needed
\end{longtable}

\subsection*{Reflectie}
\begin{longtable}{|l|p{0.8\textwidth}|}
\hline
Situatie:   & Na het lezen van de extra opdracht details van Dirk leek het erop dat het grootste gedeelte van de opdracht op de FPGA zou gebruiren wat Segers gedeelte is. \\ \hline
Taak:   & Ik wilde duidelijkheid krijgen over mijn deel van de opdracht en wat dit deel zou inhouden \\ \hline
Actie:  & Ik heb meteen een gesprek ingepland waarin Seger en ik de details van de opdracht konden bespreken op woensdag in plaats van wachten tot vrijdag. \\ \hline
Resultaat: & Het gesprek liep erg goed en uiteindelijk is mijn deel van de opdracht een stuk verduidelijkt \\ \hline
Reflectie:   & Zo snel mogelijk duidelijkheid vragen en initiatief nemen waren dingen waar ik in mijn vorige stage veel moeite mee had. Ik was dus erg tevreden met het resultaat van de meeting en mijn acties.\\ \hline

\hline

% Continue adding rows as needed
\end{longtable}

\subsection*{Reflectie-onderwerpen}
Hieronder vind je een lijst van voorbeeldonderwerpen. Je kunt wekelijks een onderwerp uit deze lijst kiezen, maar je kunt uiteraard ook een eigen onderwerp uitwerken.

\begin{itemize}
    \item Welke eerste indruk hebben collega’s van jou en waarom? Wat kun je hiervan leren voor de toekomst?
    \item Tegen welke uitdagingen ben je deze week aangelopen? Hoe heb je dat aangepakt? Wat is het resultaat? Wat heb je hiervan geleerd?
    \item Waar ben je deze week tevreden over en waarom? Wat kun je hiervan leren voor de toekomst?
    \item Lig je nog op planning? Of is het nodig om deze bij te stellen? Hoe komt dit?
    \item Werk je nog steeds toe naar de projectdoelstelling uit je PvA? Of moet deze aangepast worden?
    \item Zou je bij dit bedrijf willen werken? Waarom?
    \item Wanneer vind je iemand een goede collega? Welke van die eigenschappen herken je in jezelf en je begeleider?
    \item Hoe divers is dit bedrijf (denk aan bv. leeftijd, geslacht, afkomst, etc.)? Wat betekent dat voor de dagelijkse gang van zaken en voor jouw functioneren? Wat zou hierin ideaal zijn en waarom?
    \item Hoe duurzaam is het bedrijf? In welke opzichten kan het bedrijf hierin verbeteren?
    \item Als je dit project over mocht doen, wat zou je dan anders aanpakken en waarom?
\end{itemize}

\end{document}
